\documentclass[11pt, oneside]{article} 
\usepackage{geometry}
\geometry{letterpaper} 
\usepackage{graphicx}
	
\usepackage{amssymb}
\usepackage{amsmath}
\usepackage{parskip}
\usepackage{color}
\usepackage{hyperref}

\graphicspath{{/Users/telliott/Github-Math/figures/}}
% \begin{center} \includegraphics [scale=0.4] {gauss3.png} \end{center}

\title{Pythagorean Theorem}
\date{}

\begin{document}
\maketitle
\Large

%[my-super-duper-separator]

\label{sec:Euclid_I_47}

My favorite proof of the Pythagorean theorem relies on the construction below, sometimes called the ``bridal chair" or the ``windmill", where the central triangle is a right triangle, and the other figures are squares (Euclid I.47).  

\begin{center} \includegraphics [scale=0.40] {Pyth_new_0.png} \end{center}

We have the squares on three sides of a right triangle.

What we will show is that the rectangular area which is part of the large square, in gray, is equal in area to the entire small square, in blue.

\subsection*{preliminary}

We begin by restating a fundamental idea about the area of triangles, which is that, if two triangles have the same base and the same height, they have the same area.  So if we imagine sliding the top vertex of a triangle along a line parallel to the base, the area will not change.

\begin{center} \includegraphics [scale=0.5] {pyth11.png} \end{center}

\emph{Proof}.

The next figure shows this principle as it comes up in the proof.  The gray triangle with the black base is one-half the area of the small square.

\begin{center} \includegraphics [scale=0.3] {pyth12.png} \end{center}

Slide the vertex down to the right and the resulting triangle will still have the same area.

We can do the same thing with a triangle in the large square, below.  

The gray triangle has half the area of the part of the large square that is to the left of the dotted line, because its base is equal to the side of the square and the height extends to the right to the dotted line.  

\begin{center} \includegraphics [scale=0.3] {pyth13.png} \end{center}

Now slide the vertex up.  The area is unchanged.

Last, we observe that the two triangles have exactly the same shape.  Just rotate one to obtain the other.

\begin{center} \includegraphics [scale=0.4] {pyth14.png} \end{center}

This completes our informal proof.  

$\square$

The formal approach follows, based on triangle congruence.

\subsection*{main}

We label some points as shown:

\begin{center} \includegraphics [scale=0.25] {Pyth_new_1.png} \end{center}

As noted above, the figure is divided in half by the vertical down from $C$ to $DE$.  We will show that the area of the square $ACGF$ is equal to the area of the rectangle $ADZX$.

The proof proceeds in stages:  

$\circ$ \ (i) We repeat the observation from above that the area of a triangle does not change if we slide the upper vertex along a line parallel to the base.  So $\triangle FAC = \triangle FAB$ in area, and the same for $\triangle DAC = \triangle DAX$.

$\circ$ \  (ii) We will show that $\triangle BAF \cong \triangle DAC$.

$\circ$ \ (iii) Finally, the areas of the square and rectangle are twice that of the triangles formed by a diameter and two adjacent sides.  So for example, $\triangle FAC$ has one-half the area of the square $FACG$, and $\triangle DAX$ has one-half the area of the rectangle $ADZX$.

Parts 1 and 3 have been covered in such detail that we can accept them at this point.  The second claim remains.

\begin{center} \includegraphics [scale=0.25] {Pyth_new_2.png} \end{center}

$\circ$ \  (ii) $\triangle BAF \cong \triangle DAC$.

These two triangles have same side lengths, namely $AF = AC$ and $AB = AD$.  Let us consider the angles in between.  One triangle has $\angle BAF$ and the other has $\angle DAC$.

But these two angles share the central part $\angle BAC$, with one right angle, $\angle CAF$ added to make the first, and another right angle, $\angle BAD$ added to make the second.  

Thus, the two angles are equal, so the two triangles are congruent by SAS, Euclid I.4.  

Hence we have proved that the two colored areas in this figure are equal:

\begin{center} \includegraphics [scale=0.25] {Pyth_new_2.png} \end{center}

Finally, we could proceed to do the same thing on the right side of the figure, but we just appeal to symmetry.  All the equivalent relationships will hold.  

Addition yields the final result.

$\square$

One detail worth pointing out:  $CD \perp FB$.  This proof is left as an exercise.

\subsection*{Converse}

\label{sec:Euclid_I_48}


\subsection*{converse of Pythagorean theorem}

\label{sec:Pythagorean_theorem_converse}

In reasoning deductively, we move from the premise or premises (collection of facts, data given, previous theorems that were proved), and use logic to reach a conclusion.  The question arises whether, if we know only that the conclusion is true, does it follow logically that the premises are true?  This is the problem of the \emph{converse} of a theorem.

It may be so, or it may not.

We can state the converse of the Pythagorean theorem as follows:  suppose we have triangle such that $a^2 + b^2 = c^2$.  Does it follow that the angle between $a$ and $b$ is a right angle?

We profess to not know whether it is or is not a right angle.

\begin{center} \includegraphics [scale=0.35] {Pyth_converse.png} \end{center}

\emph{Proof}.

Suppose we know that $\triangle ABC$ has its sides such that
\[ AB^2 + BC^2 = AC^2 \]

Draw $BD$ at right angles to $BC$, and extend it to $D \mid BD = AB$.

We have $BC$ shared.

Since $\angle CBD$ is right, by the forward theorem we have 
\[ BC^2 + BD^2 = CD^2 \]
\[ BC^2 + AB^2 = CD^2 \]

But we're given
\[ BC^2 + AB^2 = AC^2 \]

It follows that the $AC = DC$, so the two triangles are congruent by $SSS$, which means that $\angle ABC = \angle DBC$ is a right angle.

$\square$

\subsection*{Quorra's corollary}

\label{sec:Quorra}

\begin{center} \includegraphics [scale=0.5] {pyth_corollary2.png} \end{center}

Let $\triangle ABC$ be \emph{any} triangle (here it is obtuse).  Draw $CM$ and $CN$ so that the new angles $\angle CMB$ and $\angle CNA$ (labeled $\phi$), are equal to $\angle C$.  The corresponding triangles are similar to the original, because they share $\phi$ plus one other from the original triangle.

We use single letters for the sides to make the algebra simpler.  $a$ is opposite both $\angle A$ and $\angle CMB$, while $b$ is opposite both $\angle B$ and $\angle CNA$, and $c$ is opposite $\angle C$.

The shortest side in $\triangle CMB$ is $x$ and the longest is $a$, while in $\triangle ABC$ the corresponding sides are $a$ and $c$.  So by equal ratios of sides in similar triangles we have that $x/a = a/c$.  The middle side in $\triangle CNA$ is $y$ and by similar logic we have that $y/b = b/c$.

\[ a^2 = cx, \ \ \ \ \ \ b^2 = cy \]
\[ a^2 + b^2 = c(x + y) \]

The sum of the squares of the two short sides of a triangle is equal to the product of the third side with the the sum of the two components $x + y$, when they are drawn with the angle $\phi$ as specified.
\begin{center} \includegraphics [scale=0.5] {pyth_corollary2.png} \end{center}
This is actually a generalization of our original algebraic proof of the Pythagorean theorem.

In the case where the angle at vertex $C$ is a right angle, then $M$ coincides with $N$, because there is only one vertical to a line from a given point.  So then and $x + y = c$, and this reduces to the Pythagorean theorem.

There are a large number of proofs of the Pythagorean theorem.  Many of them are collected or linked here:

\url{https://www.cut-the-knot.org/pythagoras/}

\end{document}