\documentclass[11pt, oneside]{article} 
\usepackage{geometry}
\geometry{letterpaper} 
\usepackage{graphicx}
	
\usepackage{amssymb}
\usepackage{amsmath}
\usepackage{parskip}
\usepackage{color}
\usepackage{hyperref}

\graphicspath{{/Users/telliott/Github-Math/figures/}}
% \begin{center} \includegraphics [scale=0.4] {gauss3.png} \end{center}

\title{Pythagorean Theorem:  Pappus}
\date{}

\begin{document}
\maketitle
\Large

%[my-super-duper-separator]

Pappus came up with a beautiful theorem which includes the Pythagorean theorem as an extension.  First, we need a simple lemma about parallelograms.

\emph{Lemma}.

Given two parallel lines:  $AD \parallel EBFC$.

If two parallelograms $AEFD$ and $ABCD$ have opposite sides on the two lines, and those two segments are of equal length, then they have equal areas.

\begin{center} \includegraphics [scale=0.25] {Pappus_pgram0.png} \end{center}

Here is a special case where they have the same base $AD$.

We proved earlier that the perpendicular between two parallel lines is of equal length no matter where it is drawn.

So the two parallelograms are each composed of pairs of triangles, which, having the same base and the same altitude, also have equal area.

$\square$

One might also just invoke Euclid I.35.

\begin{center} \includegraphics [scale=0.18] {EI_35.png} \end{center}

\subsection*{Pappus's parallelogram theorem}

\label{sec:PProof_Pappus}

In the figure below, on the two sides of any $\triangle ABC$ draw any parallelograms $ACDE$ and $BCFG$.  Extend the two new sides to meet at $H$.

Draw $HC$ and its extension such that it cuts $AB$ at $T$ and then let $HC = TU$.

Draw $AJ \parallel HCTU \parallel BI$.

\begin{center} \includegraphics [scale=0.22] {Pappus_pgram1.png} \end{center}

\emph{Proof}.

The new parallelogram with $AC$ as one side and $RH$ the other, is equal, by our lemma.  $(ACDE) = (ACHR)$.  

Since $RAJ \parallel HCTU$ and $RA = HC = TU = AJ$, $(ATUJ) = (ACHR)$ for the same reason.  

So $(ATUJ) = (ACDE)$.

We use the same argument for $(BCFG) = (TBIU)$ on the right.

Then add the two results:  $(ACDE) + (BCFG) = (ABIJ)$.

$\square$.

Let $\angle ACB$ be a right angle, and let the parallelograms be squares.

\begin{center} \includegraphics [scale=0.30] {Pappus_pgram2.png} \end{center}

$\triangle DHC$ and $\triangle FHC$ are right triangles and they are congruent to $\triangle ABC$ by SAS.

So $\angle CHF = \angle CAT$ and we also have vertical angles, so $\triangle HFC \sim \triangle CTA$.

It follows that $\angle HFC = \angle CTA$ and both are right angles.

Further, $CH = TU = AB$, so $ABIJ$ is the square on $AB$.

It is easy to see that $(AEDC) = (ARHC) = (ATUJ)$, and the rest follows.

Thus Pappus' theorem becomes the Pythagorean theorem as a special case.

[ Reference:  George F. Simmons, \emph{Calculus Gems}. ]


\end{document}