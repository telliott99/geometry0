\documentclass[11pt, oneside]{article} 
\usepackage{geometry}
\geometry{letterpaper} 
\usepackage{graphicx}
	
\usepackage{amssymb}
\usepackage{amsmath}
\usepackage{parskip}
\usepackage{color}
\usepackage{hyperref}

\graphicspath{{/Users/telliott/Dropbox/Github-math/figures/}}
% \begin{center} \includegraphics [scale=0.4] {gauss3.png} \end{center}

\title{Inscribed angles}
\date{}

\begin{document}
\maketitle
\Large

%[my-super-duper-separator]

\subsection*{Thales' circle theorem}

\label{sec:Thales_theorem}

In this chapter, we will introduce the inscribed angle theorem.  Let's start by revisiting Thales' circle theorem:

$\bullet$  Any angle inscribed in a semicircle is a right angle.

Take a diameter of the circle and any third, distinct point.  The three points on the circumference of the circle form a triangle, and the angle at the third vertex is always a right angle.

In the figure below, $\angle ABC$ is a right angle.

\begin{center} \includegraphics [scale=0.14] {EIII_20a.png} \end{center}

\emph{Proof}.

By I.32, the sum of angles in a triangle is equal to two right angles.  So
\[ \angle OAB + \angle OBA + \angle OBC + \angle OCB = 180 \]

But $\triangle OAB$ and $\triangle OBC$ are both isosceles, so the base angles are equal, with
\[ \angle OAB = \angle OBA \ \ \ \ \ \ \ \ \angle OBC = \angle OCB \]

It follows that 
\[ 2 \angle OBA + 2 \angle OBC = 180 \]
\[ \angle OBA + \angle OBC = 90 \]

$\square$

According to Boyer, this result was known to the Egyptians 1000 years before Thales.  But it is yet another example of knowing a result before proving that it is so.  Archimedes had something to say about the importance of having discovered a fact, before finding a way to prove it.

The converse of Thales' theorem is also true.  If the third point of a triangle contains a right angle, then it must lie on the circle where the other two points form the diameter.

\subsection*{Thales' circle theorem converse}
A nice direct proof of this is given in Acheson.

\label{sec:Thales_circle_theorem_converse}

\emph{Proof}.

\begin{center} \includegraphics [scale=0.4] {Acheson_G59.png} \end{center}

We are given that $\angle APB$ is a right angle.  

Draw $OD$ parallel to $PB$.  $\triangle AOD$ is similar to $\triangle ABP$ because they are both right triangles with a shared vertex at $A$.  

Since $AO$ is one-half $AB$, the scale factor is $1/2$.  In particular, $AD = DP$.

Now draw $OP$.  The two smaller triangles $\triangle AOD$ and $\triangle DOP$ are congruent by SAS.  Therefore, $OP = OA$.  

But $OA$ is a radius of the circle.  Therefore, $OP$ is also a radius.

It follows that $P$ must lie on the circle.

$\square$


We can use Thales' theorem to introduce the inscribed angle theorem.

\begin{center} \includegraphics [scale=0.14] {EIII_20a.png} \end{center}

\subsection*{inscribed angle theorem}

\label{sec:inscribed_angle_theorem}

An inscribed angle is an angle whose vertex lies on the circle, such as $\angle BAC$

$\bullet$  An inscribed angle is one-half the corresponding central angle lying on the same arc, $\angle BOC$.  The central angle is twice the corresponding inscribed angle.

\emph{Proof}.

Euclid's proof (the theorem is proposition 20 of book III), uses the external angle theorem.   $\angle BOC$ is the external angle to $\triangle OAB$.

Thus, it is equal to the sum of $\angle OAB + \angle OBA$.  

The triangle is isosceles, so these two angles are equal. 

It follows that $\angle BOC = 2 \angle BAC$.

$\square$

This proof is short and sweet, but limited by the fact that we used the diameter for one of the arms of the inscribed angle.  Here is a more general proof.

\begin{center} \includegraphics [scale=0.14] {EIII_20b.png} \end{center}

The claim is that $\angle BOD = 2 \angle BAD$.

\emph{Proof}.

Just add the results obtained using the previous proof.  $\angle DOC = 2 \angle DAC$ and $\angle BOC = 2 \angle BAC$.

\[ \angle BOD = \angle BOC + \angle DOC \]
\[ = 2 \angle BAC + 2 \angle DAC = 2 \angle BAD \]

$\square$

The proof is \emph{still} limited, since the angle we looked at includes the center of the circle.  There are two ways to fix this.  The first is subtraction.

\begin{center} \includegraphics [scale=0.14] {EIII_20c.png} \end{center}

\emph{Proof}.

By the first proof, $\angle BOD = 2 \angle BAD$ and $\angle COD = 2 \angle CAD$.  The angle of interest, $\angle BOC$, is the difference:

\[ \angle BOC = \angle BOD - \angle COD \]
\[ = 2 \angle BAD - 2 \angle CAD = 2 \angle BAC \]

$\square$

An elegant proof of the second case is as follows.

\begin{center} \includegraphics [scale=0.3] {inscribed1.png} \end{center}

\emph{Proof}.

We claim that angle $\phi$ is twice the inscribed angle $\theta$ on the same arc.

Labeling additional angles $\alpha$ and $\beta$, we see two triangles with angles $\phi$ and $\theta$ that share vertical angles.  So by sum of angles and vertical angles we have that
\[ \phi + \alpha = \theta + \beta \]

But $\triangle OAB$ is isosceles with
\[ \beta = \alpha + \theta \]
\begin{center} \includegraphics [scale=0.3] {inscribed2.png} \end{center}

By addition, canceling $\alpha + \beta$, we obtain the result:
\[ \phi = 2 \theta \]

$\square$

\subsection*{angles on the same arc}

\label{sec:angles_on_same_arc}

$\bullet$  Angles that lie on the same arc or are subtended by the same chord in the same circle, are equal.  

This theorem is Euclid III.21.  

The previous \hyperref[sec:inscribed_angle_theorem]{\textbf{inscribed angle theorem}} is Euclid III.20, which says that the central angle on any chord or arc or ``segment'' of a circle, is twice an arbitrary peripheral (inscribed) angle on the same segment of the same or equal circle.

Euclid III.21  is an immediate consequence of the previous one, because the peripheral angle in III.20 is \emph{arbitrary}.

The two theorems are sometimes thought to be the same.  III.20 might be better referred to as the central angle theorem, and III.21 as the inscribed angle theorem.  However, that is not the customary usage.

III.21 will appear often in the pages to come.  We will refer to it as \hyperref[sec:angles_on_same_arc]{\textbf{equal angles $\iff$ equal arcs}} or more simply, equal angles \emph{on} equal arcs, unless we slip, and call it the inscribed angle(s) theorem.

\begin{center} \includegraphics [scale=0.15] {inscribed angles.png} \end{center}

\emph{Proof}.

This is an immediate consequence of the previous theorem, since two such angles are equal to one-half of the \emph{same} central angle. $\angle POQ$ is the central angle for arc $PQ$ and hence is twice both $\angle PRQ$ and $\angle PSQ$.  Thus, $\angle PRQ = \angle PSQ$.

$\square$

\subsection*{converse}

\label{sec:inscribed_angles_converse}

\label{sec:equal_angle_on_circle_contradiction}

There are several variant arguments from similar setups which are converses of the inscribed angles theorem.  Here is one:

Let $\triangle ABC$ lie on a circle.  

Let point $D$ be such that $\angle BDC = \angle BAC$.  

\begin{center} \includegraphics [scale=0.16] {Coxeter_1_9_3_c.png} \end{center}

Then $D$ is also on the same circle.

\emph{Proof}.

Aiming for a contradiction, suppose otherwise.

Let $D$ be external and also let $\angle BDC = \angle BAC$.

Find where $BD$ cuts the circle at $D'$.

So $D'$ is on the circle and $\angle BD'C$ is subtended by $BC$.

By the forward version of the inscribed angle theorem:  
$\angle BD'C = \angle BAC$.

It follows that $\angle BDC = \angle BD'C$.

But $\angle BD'C$ is external to $\triangle CDD'$.

So $\angle BD'C > \angle BDC$.

This is a contradiction.

A similar argument will show that $D$ cannot be internal to the circle.

Therefore, it must be that $D$ \emph{is on the circle}.

$\square$

\subsection*{extended altitude}

\label{sec:extended_altitude}

$\bullet$ \ \ An altitude extended to the circumcircle of a triangle forms congruent triangles.

\emph{Proof}.

\begin{center} \includegraphics [scale=0.45] {altitudes2b.png} \end{center}

In $\triangle ABC$ draw the altitudes including $AD$ and $CF$ and find the orthocenter.  Draw the circumcircle an extend $AD$ to meet the circle at $X$..  

The angles at $D$ and $F$ are right.  Since they are both right triangles and share vertical angles we have
\[ \triangle AFH \sim \triangle CDH \]

So $\angle DCF = \angle FAD$

But $\angle BAX$ cuts the same arc.  Therefore the three angles marked with red dots are all equal.

$\triangle AFH \cong \triangle AFX$ (ASA).

It follows that $FX = FH$ and $\triangle GAX$ is isosceles as well.  $\triangle ABH \cong \triangle ABX$ by SAS.

$\square$

\begin{center} \includegraphics [scale=0.45] {altitudes2b.png} \end{center}

\subsection*{problem}

\begin{center} \includegraphics [scale=0.4] {Posamentier_mod_1.png} \end{center}

Posamentier gives this problem.  Let $O$ be the center of the circumcircle of $\triangle ABC$.

Let $CD \perp AB$ and $CX$ bisect $\angle C$.

Show that $CD$ also bisects $\angle ACX$.

\emph{Proof}.

Given $\angle ACX = \angle BCX$.

Draw $OC$.

The central $\angle BOC$ is 2 $\angle CAB$.

But $\triangle BOC$ is isosceles with equal base angles.

Thus $\angle CAB$ and $\angle OCB$ are complementary.

$\angle ACD$ is also complementary to $\angle CAB$, hence $\angle CAB$ is equal to $\angle OCB$.

\begin{center} \includegraphics [scale=0.4] {Posamentier_mod_1.png} \end{center}

Subtracting equals:  $\angle BCX = \angle OCX$.

$\square$

The angle bisector at $C$ also bisects the angle between the altitude and the radius from $C$.

\emph{Proof}.  (Alternate).

I have redrawn the figure slightly.

\begin{center} \includegraphics [scale=0.4] {Posamentier_mod_2.png} \end{center}

Claim: the altitude and diameter of the circumcircle through $C$ form equal angles with the sides $CA$ and $CB$.

Let $\triangle ABC$ have its circumcircle on center $O$, so $CO$ is a radius.  Extend $CO$ to form the diameter $CF$.

Let $CD$ be the altitude from $C$, so it cuts $AB$ at a right angle and extends to the circle at $E$.

By Thales' circle theorem, $\angle CEF$ is right.  It follows that $EF \parallel AB$.

Parallel chords in a circle cut equal arcs.  (Note:  we prove this elsewhere.  Hint:  connect the opposing ends of the two chords to give equal inscribed angles).

Hence $AE = BF$ and then by inscribed angles, $\angle ACE = \angle BCF$.

$\square$

\begin{center} \includegraphics [scale=0.40] {Posamentier_mod_2.png} \end{center}

By subtraction, the bisector also bisects the angle between the altitude and the radius AO.

Alternatively, find $Y$ as the midpoint of arc $AB$.  

Then we have equal arcs $AY = BY$, $AE = BF$, and $EY = FY$.  Results about the angles follow easily.

\subsection*{geometric mean}

We showed in the chapter on the \hyperref[sec:pythagorean_thm]{\textbf{Pythagorean theorem}} that the altitude of a right triangle is the geometric mean of the two components of the base.

\[ h^2 = pq \]
\[ h = \sqrt{pq} \]

According to wikipedia:

\url{https://en.wikipedia.org/wiki/Geometric_mean}

The fundamental property of the geometric mean is that (letting $m$ be the \emph{geometric mean} here):
\[ m \ [ \ \frac{x_i}{y_i} \ ] \ = \frac{m(x_i)}{m(y_i)} \]

and one consequence is that

\begin{quote}This makes the geometric mean the only correct mean when averaging normalized results; that is, results that are presented as ratios to reference values.\end{quote}

A number of examples are given in the article.

We discuss this here because originally, there was a proof-without-words that the geometric mean is always less than or equal to the arithmetic mean.

I decided to add some words.

\begin{center} \includegraphics [scale=0.4] {arcs15b.png} \end{center}
A right triangle is inscribed in a semicircle.

We can use the Pythagorean theorem three times or just rely on the fact that the two smaller triangles are similar with equal ratios of sides including:

\[ \frac{h}{a} = \frac{b}{h} \]
and the result follows immediately:
\[ ab = h^2 \]

This says that the square of the altitude $h$ is equal to the product of chord segments (we will prove this geometrically in the next chapter as well).

\[ h = \sqrt{ab} \]

\begin{center} \includegraphics [scale=0.4] {arcs15b.png} \end{center}

But we also have that $a + b = 2r$ and hence
\[ r = \frac{a + b}{2} \]

Do you recognize these?  The second expression is the arithmetic mean of $a$ and $b$, while the first is the geometric mean.

The geometry shows that $h \le r$ so:
\[ \sqrt{ab} \le \frac{a + b}{2} \]

The geometric mean is always less than the arithmetic mean, except when $a = b$, where they are equal (or all of $n$ values are equal, when there are more than two values).

\subsection*{problem}

I have lost track of where I found this problem.

Given $\triangle ABC$ with its circumcircle.  Given $AD$ is an altitude.

Extend $AD$ to meet the circle at $X$, and also extend $AD$ vertically to $H \mid HD = DX$.  Join $HB$ and extend $CA$ to $E$.  Show that $CAE\perp HB$.

\begin{center} \includegraphics [scale=0.4] {inscribed_problem.png} \end{center}

Hint:  draw point $P$ where $HB$ cuts the circle.  Show that arc $PAC = $ arc $CX$.

Ignoring the hint for now, we are given that $BC$ is a perpendicular bisector of $HX$.  

Hence $\triangle HBX$ is isosceles with $HB = BX$ and $\angle BXD = \angle H$.

But $\angle BXD$ and $\angle C$ cut the same arc $AB$.  So $\angle H = \angle C$.

We also have vertical angles so it follows that $\triangle AEH \sim \triangle ADC$.  Thus $\angle AEH$ is right.  

$\square$

Going back to the hint about $P$.  Again, $\triangle HBX$ is isosceles.  

\begin{center} \includegraphics [scale=0.4] {inscribed_problem.png} \end{center}

$\angle PBC = \angle XBC$, so arc $PAC = XC$.

Then $\angle HAE = \angle DAC$ and both equal $\angle PBC$ since they stand on equal arcs.

$\angle H$ is shared.  It follows that $\triangle HAE \sim \triangle EBC$.  Since $BPEH$ are collinear, and the angles at $E$ are equal, they are both right.

\subsection*{problem}

\label{sec:sec_tan_problem}

Here is a problem I found on the web as a Youtube video:

\url{https://youtu.be/2Jt8lynddQ8}

It is described as a GCE O-Level A-Maths Plane Geometry Question.  

The relationships that seem obvious from the diagram are given.  Namely, $PXYR$ and $QXZS$ are each lie on a straight line (colinear).

And the two circles each have the four points lying on them as shown.  

$TPS$ is tangent to the smaller circle at $P$.
\begin{center} \includegraphics [scale=0.4] {prob_A_level1a.png} \end{center}

The problem asks us to show that $SR$ is parallel to $ZY$ and hence, \emph{deduce} that $YX/ZX = YR/SZ$.

The approach that first occurred to me was to use the similar triangles that arise from crossed chords.  However, the problem asks us to start by showing that the given line segments are parallel.  This is a hint to a different proof.

The result comes from the theorem which is the basis of this chapter: the inscribed angle theorem.
\begin{center} \includegraphics [scale=0.4] {prob_A_level1b.png} \end{center}

\emph{Proof}.

The marked angles are all equal.  The first two are equal because they correspond to the same arc in the small circle, and the third (at $S$) is equal to the first because they both correspond to the same arc in the large circle.  

Therefore, the two line segments are parallel by the converse of the alternate interior angles theorem.

That gives us similar triangles $\triangle XYZ \sim \triangle XRS$ from which the equal ratios follow almost immediately (see below).

$\square$

The last part of the problem says that given $SQ = XR$, prove that $PS^2 = XS \cdot YR$  We're not ready to do that yet.  It uses the information about the tangent and the \hyperref[sec:tangent_secant_theorem]{\textbf{tangent-secant theorem}}.

\emph{Proof}.  (Alternate).

Here's the first part of the proof by the crossed chord theorem:
\[ PX \cdot XY = QX \cdot XZ \]
\[ PX \cdot XR = QX \cdot XS \]

It follows that
\[ \frac{XZ}{XY} = \frac{XS}{XR} \]

We need some algebra to get to $YX/ZX = YR/SZ$.  This is a standard parts and the whole manipulation for similar triangles, made complex by the cumbersome notation.

Let $a = XY$, $A = XR$, $b = XZ$, and $B = XS$.  We have
\[ \frac{b}{a} = \frac{B}{A} \]
and we want
\[ \frac{a}{b} = \frac{A-a}{B-b} \]

The way to get there is:
\[ \frac{A}{a} = \frac{B}{b} \]
\[ \frac{A}{a} -1 = \frac{B}{b} - 1 \]

then
\[ \frac{A-a}{a} = \frac{B - b}{b} \]
and the result follows easily in one more step.

$\square$

\subsection*{problem}

\begin{center} \includegraphics [scale=0.3] {circles1.png} \end{center}

Two circles meet at $Q$ and $S$.  $QR$ and $QT$ are diameters of the two circles.  Prove that $RST$ are colinear.

\begin{center} \includegraphics [scale=0.3] {circles2.png} \end{center}

Since $QR$ is a diameter of the circle centered at $O$, $\angle QSR$ is a right angle.  

But so is $\angle QST$, since $QT$ is a diameter of the second circle.  

Hence the total angle at $S$ is two right angles or a straight line.  Therefore $RS$ and $ST$ togethere form a straight line segment.

\subsection*{double arc problem}

This problem is taken from an online collection by David Surowski.

\url{https://www.math.ksu.edu/~dbski/writings/further.pdf}

\begin{center} \includegraphics [scale=0.6] {further1.png} \end{center}

Given that $AB$ and $AC$ are tangents to the circle meeting at $A$.  Given a second tangent $DE$, meeting the circle at $P$.  Prove that the arc that subtends $\angle BOC$ is twice that which subtends $\angle DOE$.

\emph{Proof}.

Notice that $DB$ and $DP$ are tangents to the circle meeting at $D$.  Therefore $DB = DP$ and then $\triangle BOD \cong \triangle DOP$, so $\angle BOD = \angle DOP$.

The same argument applies to $EC$ and $EP$.  Therefore the inner arc is composed of two angles, while the outer arc has two copies of each of those angles.

$\square$


\subsection*{Queen Dido}

The mighty city of Carthage was the capital city of a major Phoenician colony.  As Rome grew strong, there was a titanic struggle between the two peoples, which Carthage eventually lost.  The ruins of Carthage lie near present-day Tunis.

Queen Dido was the legendary founder of the the city of Carthage.  She was supposedly 

\begin{quote}granted as much land as she could encompass with an oxhide.  She promptly cut the ox-hide into very thin strips.\end{quote}

The problem then is to maximize the area enclosed by a curve of fixed length.

\begin{center} \includegraphics [scale=0.5] {Dido.png} \end{center}

In calculus there are a number of problems like this.  What's nice is that this problem has a wonderful solution that uses only the tools we have so far.  In particular, we need the converse of Thales' circle theorem.

The argument goes as follows.  Suppose we have a particular outline for the city limits and we're pretty happy with it.  We suppose it is a maximum (left panel).

\begin{center} \includegraphics [scale=0.5] {Dido2.png} \end{center}

Then we notice that by rearranging $AD$ and $BD$ so they meet at a right angle, the crescent-shaped areas are unchanged, but the area of $\triangle ABD$ is a maximum.  That's because a right triangle, having the two sides at right angles, has area equal to the product of the two sides (divided by $2$).  No other triangle with the same two sides has as much area.

So the arrangement on the right has a bigger area.

But then, with $AB$ as the diameter of a circle, if $\angle ADB$ is a right angle, it must lie on the circumference of that circle, by the converse of Thales' theorem.

And this is true regardless of the relative lengths of $AD$ and $BD$.  Therefore the maximum area is obtained when $D$ traces out a semi-circle.

This example is in Acheson's Geometry.


\end{document}