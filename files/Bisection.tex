\documentclass[11pt, oneside]{article} 
\usepackage{geometry}
\geometry{letterpaper} 
\usepackage{graphicx}
	
\usepackage{amssymb}
\usepackage{amsmath}
\usepackage{parskip}
\usepackage{color}
\usepackage{hyperref}

\graphicspath{{/Users/telliott/Dropbox/Github-Math/figures/}}
% \begin{center} \includegraphics [scale=0.4] {gauss3.png} \end{center}

\title{Constructions}
\date{}

\begin{document}
\maketitle
\Large

%[my-super-duper-separator]

Here we look at the problem of angle bisection, cutting an angle into two equal parts.  We also look at the perpendicular bisector of any line, a perpendicular line that cuts halfway between two end points, or a vertical through any particular point on a line, or third, through a point not on the line.  

We will show that every point on the perpendicular bisector is equidistant from the two points used to construct it.

We also prove the converse theorem, that \emph{every} point which is equidistant lies on the bisector.

\subsection*{constructions}

A number of the propositions in book 1 of Euclid concern constructions.

\begin{center} 
\includegraphics [scale=0.2] {straightedge.png} 
\includegraphics [scale=0.3] {compass.png} 
\end{center}

The tools we have are a straight-edge and a compass.  The compass is collapsible, meaning that it cannot be used to transfer distances since it loses its setting when lifted from the page.  This is a limitation Euclid solves in the second and third propositions of book I.  It's also important that the straight-edge is not a ruler, there is no way to measure distance by reference to marks on it.

Euclid was smart enough to know about compasses and how to set them.  The idea he had was this:  to make the fewest possible assumptions.  A non-collapsible compass was a luxury he didn't need, since he could accomplish the same end without it.

\subsection*{Euclid I.1:  equilateral triangle}

\label{sec:Euclid_I_1}

To construct an equilateral triangle on a given line segment.
\begin{center} \includegraphics [scale=0.3] {PI_1a.png} \end{center}

The first step is to draw two circles on centers $A$ and $B$.
\begin{center} \includegraphics [scale=0.16] {bisect1.png} \end{center}

The circles are drawn with each radius equal to the line segment $AB$.  It is a property of circles that all points on the circle are at the same distance from the center.  Thus all points on the left-hand circle are equidistant from $A$, and all points on the second one are equidistant from $B$.  

Therefore, the point $C$  where the circles cross is equidistant from \emph{both} $A$ and $B$.

Now use the straight edge to draw $\triangle ABC$.  Since $AC = AB$ and $BC = AB$, we know that $AC = BC$.  The triangle is equilateral.

\begin{center} \includegraphics [scale=0.4] {PI_1c.png} \end{center}

$\square$

The proof doesn't stand on its own.  We used one definition (D) and a common notion (CN).

$\circ$ \ D I.15  all radii of a circle are equal.

$\circ$ \ CN I.1  things which equal the same thing also equal one another.

If we look again at the figure, and label the other point where the circles cross as $D$:
\begin{center} \includegraphics [scale=0.16] {bisect2.png} \end{center}

We have a second equilateral triangle, congruent to the first.

\subsection*{Euclid I.9:  bisection of an angle}

\label{sec:Euclid_I_9}

To bisect a given angle.  Let $\angle BAC$ be the angle to bisect.

\begin{center} \includegraphics [scale=0.16] {bisect4.png} \end{center}

As radii of a circle on center $A$, we first find points $D$ and $E$ equidistant from $A$.  So $AD = AE$.

Next, we draw circles on centers $D$ and $E$ that have the same radius, and the easiest way to do that (with a collapsible compass), is to make the radius equal to $DE$.

As radii of circles on the centers $D$ and $E$, $DE = DF = EF$ and $\triangle DEF$ is equilateral.

Here is a somewhat cleaner view.

\begin{center} \includegraphics [scale=0.16] {Euclid_I_9.png} \end{center}

$AF$ is shared, so $\triangle ADF \cong \triangle AEF$ by SSS, Euclid I.8.

Therefore $\angle BAF$ is congruent to $\angle CAF$ and the given angle $\angle BAC$ is bisected.

$\square$

\subsection*{perpendicular lines}

When constructing a line segment perpendicular to another line segment, there are three common situations.  We want the perpendicular line to pass:

\begin{itemize}

\item through a given point $P$ on the line.

\item halfway between two points $Q$ and $R$ on the line, bisecting $QR$

\item through the line and also through a point $S$ not on the line

\end{itemize}

\begin{center} \includegraphics [scale=0.4] {perp8.png} \end{center}

We solve the second case and then show how the other two can be adapted to it.

\subsection*{Euclid I.10:  perpendicular bisector}

\label{sec:Euclid_I_10}

Simply construct two circles of equal radius, one centered at $Q$ and the other at $R$.  It's easiest to choose the radius to be equal to the length $QR$.

\begin{center} \includegraphics [scale=0.16] {bisect3.png}  \end{center}

As before, the point $S$ is on both circles, hence it is a radius of both, and therefore equidistant from $Q$ and $R$.  $QS = SR = QR$.  The three points form an equilateral triangle, $\triangle QRS$.

The point $T$ below the line segment has the same property, and $\triangle QRS$ is congruent to $\triangle QRT$ by SSS.

Furthermore, we claim that the angles as $P$ are right angles.  Thus, $SPT \perp QPR$.

Euclid's proof is simple. 

\begin{center} \includegraphics [scale=0.16] {bisect3.png}  \end{center}

\emph{Proof}.

$\triangle QST \cong \triangle RST$ by SSS, and both are isosceles.

It follows that $\angle QSP = \angle RSP$, i.e. $\angle QSR$ is bisected.

So $\triangle QPS \cong \triangle RPS$ by SAS.

This gives $QP = RP$.

Also, the angles at $P$, $\angle QPS = \angle RPS$, so all four are right angles.

$\square$

\subsection*{Euclid I.11:  bisector through a given point on the line}

\label{sec:Euclid_I_11}

Suppose we know a point $P$ on the line and wish to construct the vertical line through $P$.  

Use the compass to mark off points $Q$ and $R$ on both sides of $P$, equidistant from it.  This can be done by drawing a circle with center at $P$.

\begin{center} \includegraphics [scale=0.4] {perp_7.png} \end{center}

Now, simply proceed as above.

\subsection*{Euclid I.12:  bisector through a given point not on the line}

\label{sec:Euclid_I_12}

Alternatively, suppose we know the line and the point $U$ but not $P$, and we wish to construct a vertical through the line that also passes through $U$.  

Find $Q$ and $R$ on the line an equal distance from $U$ ($QU$ = $RU$), as radii of a circle centered at $U$ (left panel, below).  Their exact position is unimportant.  

\begin{center} \includegraphics [scale=0.35] {perp11.png} \end{center}

Now repeat the previous construction, using $Q$ and $R$.  The line segment $ST$ passes through $U$, as required.

\emph{Proof}.  (Sketch).  

Since $QU = RU$, $\triangle QUR$ is isosceles.  Therefore the base angles are equal.  In an isosceles triangle, the top vertex lies on the perpendicular bisector of the base  (see the chapter on isosceles triangles.

Alternatively, find a point below the line equidistant from $Q$ and $R$.  Proceed as in the proof above.

\subsection*{bisector is the altitude of an isosceles triangle}

Suppose we know two points $B$ and $C$.  We find the point $M$ equidistant between them and construct the perpendicular bisector $AM$.  Then the two sides $AB$ and $AC$ have equal length.  $\triangle ABC$ is isosceles.

\begin{center} \includegraphics [scale=0.2] {perp3b.png} \end{center}

That is true for \emph{any} point on the line extended through $A$ and $M$.  For example, $A'B = A'C$ in the figure above.

\subsection*{perpendicular bisector converse}

The converse theorem says that \emph{every} point which is equidistant from $B$ and $C$ lies on $AM$ or an extension of it.  

Here are two proofs.  The first one relies on the \hyperref[sec:triangle_inequality]{\textbf{triangle inequality}}, which says that in any triangle, the sum of any two sides must be greater than the length of the third side.

We give a plausible argument for the triangle inequality based on a construction.  (We'll look at Euclid's proof later).  The theorem is trivially true if the third side is not the longest.  Hence we compare the two shorter sides to the longest one.

\begin{center} \includegraphics [scale=0.2] {tri_ineq1} \end{center}

Let $AB$ be the points connected by the longest side $c$.  We claim that the sum of the lengths of the other two sides $a$ and $b$ must be such that:
\[ a + b > c \]

Draw two circles, one on center $A$, and the other on center $B$, with radii $a$ and $b$ the same as the two shorter sides.  There are two points where those circles cross, $C$ and $C'$.  Triangles constructed using either one of those as the third vertex are congruent by SSS.  (Mirror images are fine).

Now consider what happens as one of those two sides gets shorter.  Let $a$ be the side that is decreased in length.
\begin{center} \includegraphics [scale=0.25] {tri_ineq2} \end{center}

\begin{center} \includegraphics [scale=0.25] {tri_ineq3} \end{center}

As the combined length drops to be less than the longest side, it is no longer possible to construct a triangle.

$\square$

To the current proof:

\emph{Proof}.

Suppose that $P$ is equidistant from $B$ and $C$ but does not lie on the perpendicular bisector.  Suppose that $P$ lies on the same side of the bisector as $B$.

Find the point where $PC$ crosses the bisector at $A$.
\begin{center} \includegraphics [scale=0.16] {perp_8.png} \end{center}
By the forward theorem, $AB = AC$.

We are supposing that $PB = PC$.  By the triangle inequality
\[ PB < AB + AP \]
Since $AB = AC$:
\[ PB < AC + AP = PC \]

But this is absurd.  $PB$ cannot both be equal to and less than $PC$.  Therefore, our supposition is incorrect, and there does not exist any such point $P$.

$\square$

See \hyperref[sec:perp_bi_converse]{\textbf{here}} for a somewhat fuller explanation.

\subsection*{perpendicular bisector using I.7}

We will prove that every point which is equidistant from two points on a line lies on the perpendicular bisector.

Let $AM \perp BC$, and bisect it such that $BM = CM$.  $AM$ is the perpendicular bisector of $BC$.

\begin{center} \includegraphics [scale=0.18] {perp_9.png} \end{center}

\emph{Proof}.

Suppose $P$ lies on the same side of $AM$ as $B$.

Now, seeking a contradiction, suppose $PM$ is also perpendicular to $BC$ at $M$.

$PM$ bisects $BC$ so it is a perpendicular bisector of $BC$.

By the forward theorem $PB = PC$.

Now find $P'$ on $AM$ such that $P'B = PB$.  

We have three equal segments:  $PB = PC = P'B$.

By the forward theorem, since $P'$ is a point on the perpendicular bisector, $P'B = P'C$.

Now we have four equal segments:  $PB = PC = P'B = P'C$.

But by Euclid I.7, this is impossible.  It cannot be that $P'B = PB$ and also $P'C = PC$ on the same side of $BC$.

This is a contradiction.  There is only one perpendicular through $BC$ at $M$.

$\square$

\begin{center} \includegraphics [scale=0.18] {perp_9.png} \end{center}


\subsection*{bisector equidistant from sides}

\label{sec:bisector_equidistant_sides}

$\bullet$ \ Any point on the bisector of an angle is equidistant from the sides at the point of closest approach.

\begin{center} \includegraphics [scale=0.18] {angle_bisector2c.png} \end{center}

\emph{Proof}.

Let $PR$ bisect $\angle SPT$.

Draw perpendiculars $QS$ and $QT$.  

The angle at $P$ is bisected, so $\triangle PQS$ and $\triangle PQT$ have two angles equal, and by sum of angles they have three angles equal.

They share the hypotenuse, $PQ$.

So $\triangle PQS \cong \triangle PQT$ by ASA.

It follows that $QS = QT$.

$\square$

\subsection*{equidistant from sides $\rightarrow$ bisector}

\label{sec:bisector_equidistant_sides_converse}

$\bullet$ \ If a point is equidistant from the sides of an angle, then it lies on the angle bisector.

\emph{Proof}.

Given $QS = QT$.  We have $PQ$ shared.

So $\triangle PQS \cong \triangle PQT$ by HL.

It follows that $\angle PQS = \angle PQT$.

$\square$

\subsection*{problem}

Here is a problem to exercise some concepts we've seen to this point:  isosceles triangles, complementary angles, and vertical angles.

\begin{center} \includegraphics [scale=0.35] {tr3c.png} \end{center}

Given that $AB = AC$.  Pick any point $D$ on the base $BC$ (except directly under $A$), and draw the vertical $DF$.  Extend $AC$ to meet that vertical line.

Prove that $\triangle AEF$ is isosceles.

\subsection*{problems}

To prove:

$\circ$ \ Prove that for two supplementary angles, the angle bisectors are perpendicular to each other.

$\circ$ \ Prove that an equilateral triangle (all 3 sides equal) is equiangular (all 3 angles equal).  (Don't just rely on symmetry.  Adapt the proofs given in this chapter).

$\circ$ \ A line perpendicular to the bisector of an angle cuts off congruent segments on its sides.

In the figure below, given that $AC = AB$ and $\angle B = \angle C$.

\begin{center} \includegraphics [scale=0.4] {iso1.png} \end{center}

Prove that $BE = DC$.

Hint:  draw $BC$ and then mark all the angles that are equal.

\begin{center} \includegraphics [scale=0.4] {iso2.png} \end{center}

$\circ$ \ An equilateral triangle has all three angles equal.

\subsection*{problem}

\begin{center} \includegraphics [scale=0.4] {Hopkins_155.png} \end{center}

\begin{center} \includegraphics [scale=0.40] {iso_ext_prob_c.png} \end{center}

The angles labeled with red dots are equal by alternate interior angles and then vertical angles, while the angles labeled black are equal by alternate interior angles.

But we are given that this triangle is isosceles, so black and red are equal.  Therefore the exterior angle at $C$ is bisected by the horizontal.

Alternatively, use the fact that the external angle is the sum of the two base angles, and just use alternate interior angles once.

It is generally true that the adjacent bisectors of an internal and external angle form a right angle.  Above, $\angle MCF$ is right.

\begin{center} \includegraphics [scale=0.16] {bisector_ext2.png} \end{center}

$\angle DBF$ is right.

\emph{Proof}.

Twice $\angle DBC$ plus twice $\angle FBC$ is equal to two right angles.  

The result follows easily.

$\square$

\end{document}