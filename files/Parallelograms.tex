\documentclass[11pt, oneside]{article} 
\usepackage{geometry}
\geometry{letterpaper} 
\usepackage{graphicx}
	
\usepackage{amssymb}
\usepackage{amsmath}
\usepackage{parskip}
\usepackage{color}
\usepackage{hyperref}

\graphicspath{{/Users/telliott/Github-Math/figures/}}
% \begin{center} \includegraphics [scale=0.4] {gauss3.png} \end{center}

\title{Area}
\date{}

\begin{document}
\maketitle
\Large

%[my-super-duper-separator]

\subsection*{diagonal forms congruent triangles}

There is a close relationship between parallelograms and triangles.  The area of a triangle is one-half the area of the corresponding parallelogram.  In the case of a right triangle the corresponding figure is a rectangle.

As we've seen, a line joining opposite vertices of a rectangle or parallelogram is called a \emph{diagonal}.  Any diagonal divides a parallelogram into two congruent triangles.  The difference is that starting from a rectangle, we obtain right triangles.

\begin{center} \includegraphics [scale=0.3] {rect_pgram.png} \end{center}

$\triangle ABC \cong \triangle CDA$.

$\triangle EFG \cong \triangle GHE$, and $\triangle EFH \cong \triangle GFH$.

\emph{Proof}.

One can use SSS (opposite sides equal), or SAS, or even ASA (opposite sides parallel), 

$\square$

Conversely, two copies of the same triangle (i.e. congruent) can always be joined into a parallelogram, and that figure is a rectangle, if we start with a right triangle.

\begin{center} \includegraphics [scale=0.3] {rect_pgram.png} \end{center}

We can see each of the three possible ways of joining two identical triangles in the figure for the midpoint theorem.

\begin{center} \includegraphics [scale=0.2] {rot_triangle.png} \end{center}

One must be careful, however.  The triangles to be joined must not be mirror images, otherwise one may obtain a dart or a kite.

\begin{center} \includegraphics [scale=0.2] {dart_kite.png} \end{center}

By our fundamental definition of what area is, for a rectangle it is the product of the lengths of two adjacent sides.  The area of a parallelogram must be adjusted somehow for the fact that it isn't standing up straight.  The easiest way to deal with this is Euclid's approach.

\subsection*{Euclid I.35}

\label{sec:Euclid_I_35}

As detailed in Book I of \emph{Elements}, here is Euclid I.35.  This proposition (theorem) says that given any two parallelograms $ABCD$ and $EBCF$ on the same base $BC$ and with $ADEF \parallel BC$, they have the same area.

The diagram (below) is drawn for the awkward case, where the two parallelograms overlap.  

We have $AB = DC$, $AD = BC$, $BC = EF$ and the sides are parallel as well.

\begin{center} \includegraphics [scale=0.18] {EI_35.png} \end{center}

\emph{Proof}.

Because of the shared base, $AD = EF$.  

By addition:  $AE = DF$.  

We also have $AB = DC$ and because $AB \parallel DC$, $\angle EAB = \angle FDC$.  

Thus $\triangle EAB \cong \triangle FDC$, by SAS.

Subtract the shared area of $\triangle DGE$ and add the shared area of $\triangle GBC$ to obtain equality for the area of the two parallelograms.

$\square$

This result is immediately extended to the case where $\angle ABC = \angle DCB$ and both are right angles.

By I.35, this rectangle has the same area as the parallelogram.

Furthermore, the theorem can be invoked sequentially to prove that two parallelograms which do not overlap at all are equal in area, if they lie anywhere on equal bases between two parallel lines.

Roughly speaking, in the figure below, cut off a right triangle from the left side and attach it on the right.  The angles add up to form a straight line along the base and a right angle at the upper right.    The area is $h \cdot b$.

\begin{center} \includegraphics [scale=0.5] {area7.png} \end{center}

\subsection*{Area of a triangle}

\label{sec:triangle_area}

To reiterate, any triangle can be turned into a parallelogram, by attaching a rotated image of itself:

\begin{center} \includegraphics [scale=0.2] {rot_triangle.png} \end{center}

It does not matter which side we choose.  Any pair of triangles containing the central one is a parallelogram, since it has both pairs of opposite sides equal.  We have shown that this is sufficient to know we have a parallelogram.

In the figure below, $ACBC'$ is a parallelogram composed of two copies of $\triangle ABC$. The area of this parallelogram is twice that of $\triangle ABC$.  To obtain the value, multiply the base $BC$ times the "height" $E'H$.

\begin{center} \includegraphics [scale=0.4] {area4.png} \end{center}

$E'H$ is equal in length to the \emph{altitude} of the triangle.  That would be a line dropping vertically from $A$ and making a right angle with the base, $BC$.

The area of the triangle is one-half that of the parallelogram that contains two copies of the triangle.

\[ \mathcal{A} = \frac{1}{2} \cdot BC \cdot E'H \]

Euclid I.35 and the opposite of dissection, that every triangle can be assembled into a parallelogram that clearly has twice the area, together give everything we need.

If we go back to the idea of cutting off a triangle from one side of a parallelogram and placing it on the other side, to form a rectangle:
\begin{center} \includegraphics [scale=0.5] {area7.png} \end{center}

it is possible that a parallelogram is particularly skinny for a given height, then it will not be possible to cut off just one triangle.

There are at least three solutions to this.  One is to rotate the figure so that the sides become the base and the top.  A second idea is to slice the parallelogram into identical horizontal pieces, do the operation on each slice, and add them up the result.

\begin{center} \includegraphics [scale=0.4] {cut_parallel_crop.png} \end{center}

A sophisticated approach is to call on I.43 and change the shape to a different one with equal area.

\subsection*{Euclid I.43}

\label{sec:Euclid_I_43}

This theorem says that, in the figure below, the two smaller parallelograms not on the diagonal, $ABED$ and $EFIH$, are equal.  

\begin{center} \includegraphics [scale=0.15] {EI_43.png} \end{center}

\emph{Proof}.

$GEC$ is the diagonal for three parallelograms and cuts each into two congruent triangles with equal area, by the diagonal theorem.

Subtracting
\[ (ACG) - (BCE) - (DEG) = (ABED) \]
\[ (CIG) - (CFE) - (EHG) = (EFIH) \]

Since the left-hand sides are equal, it follows that $(ABED) = (EFIH)$.

$\square$

\subsection*{squaring figures}

Any parallelogram (such as $ABED$) can be turned into another with the same area but a different shape, simply by inclining the diameter $CG$ at an appropriate angle in theorem I.43.

Any parallelogram can be turned into a rectangle with the same area, as described in this chapter.

Any rectangle can be turned into a square with the same area.  This is Euclid II.14, which we will see later.

\subsection*{problem}

Given that $D$ and $E$ are midpoints of their sides:  $AD = DB$ and $AE = EC$.  Prove that the colored areas are equal.

\begin{center} \includegraphics [scale=0.2] {tra1b.png} \end{center}

\emph{Solution}.

Let the four triangular areas be labeled $I-IV$.  
\begin{center} \includegraphics [scale=0.2] {tra2b.png} \end{center}

Then, by the area-ratio theorem and using $AC$ as the base, and since the base is bisected:

\[ I + II = III + IV \]

But using $AB$ as the base

\[ I + III = II + IV \]

Add the two equations and cancel $II + III$ on both sides:
\[ I = IV  \]

Subtract the two equations and 
\[ II - III = III - II \]
\[ II = III \]

$\square$

By the \hyperref[sec:midline_theorem]{\textbf{midline theorem}}, since $DE$ bisects the sides it is parallel to the base $BC$.

\subsection*{twice the area}

It can be convenient to write the formula as \emph{twice} the area, and we will often do that.

\[ 2 \mathcal{A}_{\triangle} = ab \]

We rewrite two important formulas from this chapter:

\[ d \cdot h + e \cdot h = (d+e) \cdot h = c \cdot h \]
\[ \frac{\mathcal{A}_A}{\mathcal{A}_B} = \frac{ah}{bh} = \frac{a}{b} \]

The first one is an equality of different areas, and the second involves a ratio of areas on the left-hand side.  The result is unchanged by using twice the area, as long as we are consistent.

\end{document}