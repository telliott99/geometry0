\documentclass[11pt, oneside]{article} 
\usepackage{geometry}
\geometry{letterpaper} 
\usepackage{graphicx}
	
\usepackage{amssymb}
\usepackage{amsmath}
\usepackage{parskip}
\usepackage{color}
\usepackage{hyperref}

\graphicspath{{/Users/telliott/Dropbox/Github-math/figures/}}
% \begin{center} \includegraphics [scale=0.4] {gauss3.png} \end{center}

\title{Tangents}
\date{}

\begin{document}
\maketitle
\Large

%[my-super-duper-separator]

\subsection*{tangent:  perpendicular $\rightarrow$ touches one point}

\label{sec:tangent_one_point}

We show \hyperref[sec:Euclid_I_11]{\textbf{here}} that one can construct a line perpendicular to any given line, and passing through any point whether on the line or not.

If we make the construction perpendicular to the radius (or diameter) at the point where it meets the circle, the new line is called a tangent line (from the Latin \emph{tangere}, to touch).  

$\bullet$   The tangent line, defined as perpendicular to the radius, touches the circle at a single point.

\emph{Proof.}

By definition, the tangent line at $P$ is perpendicular to the diameter or radius including $O$ and $P$.  

Let $Q$ be some other point on that tangent line.  Then $\angle OPQ$ is a right angle.

\begin{center} \includegraphics [scale=0.35] {circle3.png} \end{center}

Let $T$ be on the tangent line so $QPT$ collinear.  Draw $OT$.

Aiming for a contradiction, suppose that $OT \perp QPT$.

By the parallel postulate, $OT \parallel OP$.

But $OT$ meets $OP$ at $O$.  This is a contradiction.

$OT$ cannot be perpendicular to $QPT$.

It follows that $OT$ is the hypotenuse of a right triangle $\triangle OPT$, so $OT > OP$ and thus $T$ cannot be on the circle.

$\square$

\subsection*{tangent:  touches one point $\rightarrow$ perpendicular}

\label{sec:tangent_perpendicular}

An alternative definition of the tangent is that it is a line that touches the circle at just one point.  One can use this definition to prove that the angle between the tangent and the radius is a right angle.  This is the converse of the previous theorem.

$\bullet$   The tangent line, defined as touching the circle at a single point, is perpendicular to the radius.

\emph{Proof}.

Draw the line that touches the circle at only one point $P$, and draw the radius to that point, $OP$.  

If we assume that $OP$ is not perpendicular to $QPT$, we can derive a contradiction.

Consider a succession of points moving along the line away from $P$ in either direction.  The angle formed with a line drawn through $O$ gets smaller as the points get farther from $P$.

The angle between the tangent and the radius is greater than a right angle on one side of $P$ (since the angle at $P$ is not a right angle).  On that side, move along the line until we find the point that does form a right angle with a radius of the circle.  

Let us suppose the point is $T$, in the figure above.  $OTP$ is a right angle and we are in doubt about whether $T$ lies inside or outside the circle.

\begin{center} \includegraphics [scale=0.35] {circle3.png} \end{center}

But then $OTP$ is a right angle, with the side opposite, namely, $OP$, the hypotenuse of a right triangle.  But the hypotenuse is the longest side in a right triangle, so therefore $OT < OP$, so the point $T$ is \emph{inside} the circle.

We have a line with one point on the circle, and another point inside the circle.  Any line through an interior point of a circle must cross the circle at two points, which contradicts the assumption above.  Hence the angle at $P$ is a right angle.

$\square$

\subsection*{shortest distance to the circle}

Let $A$ be any point inside a circle.  Draw the radius that passes through $A$ to point $P$ on the circle.  I claim that the length $AP$ is smaller than the distance to \emph{any} other point on the circle, such as $Q$.
\begin{center} \includegraphics [scale=0.35] {tangent3d.png} \end{center}

\emph{Proof}.

Draw the radius $OQ$, which is equal to radius $OP$.
\[ OQ = OP = OA + AP \]

By the \hyperref[sec:triangle_inequality]{\textbf{triangle inequality}}
\[OA + AQ > OQ \]

so
\[OA + AQ > OA + AP \]
\[ AQ > AP \]

$\square$

\subsection*{construction of a tangent}

\label{sec:tangent_construction}

\hyperref[sec:Thales_theorem]{\textbf{Thales theorem}} provides a way to construct the tangent to a circle that passes through any exterior point $P$ --- actually, there are two such lines.

\begin{center} \includegraphics [scale=0.35] {tangent1.png} \end{center}

Use OP as the diameter of a circle.  Draw the line segment $OP$ and divide it in half by erecting the perpendicular bisector at $Q$.  Use that point $Q$ as the center of a new circle with radius $OQ$.  The point $T$ is the intersection of the two circles.

\begin{center} \includegraphics [scale=0.35] {tangent2.png} \end{center}

By the theorem, $\angle OTP$ is a right angle, and since $OT$ is a radius of the original circle, $TP$ is the tangent to the smaller circle at the point $T$.

To construct a tangent on a circle at a given point $T$:

\begin{center} \includegraphics [scale=0.35] {tangent3.png} \end{center}

Extend $OT$ to $P$ so that $OP$ is twice the radius $OT$.  Construct the perpendicular bisector at $T$.  The bisector is also the tangent of the circle.

$\bullet$   From any external point $P$, one can draw two tangents to a circle.  These two tangents have the same length.

$\bullet$   From any external point $P$, the line to the center of the circle bisects the angle between the two tangents, as well as the angle between the radii drawn to the two points of tangency.

\begin{center} \includegraphics [scale=0.35] {tangent9.png} \end{center}

\emph{Proof.}

The angle between a tangent and the radius to the point where it touches the circle, is a right angle.  For a pair of tangent lines from a given point, there are two such points on the circle.

The base lengths are both radii, so they are equal, and there is a shared side (the dotted line segment $OP$).  

Therefore the two triangles are congruent, by hypotenuse-leg in a right triangle (HL).

The two congruent sides $a$ and $b$ are the same length.

\[ a = b \]

For any point external to a circle, two tangents to the circle can be drawn, of equal length.  The line from the point to the center of the circle bisects the angle between the two tangents.

$\square$

\subsection*{tangent-chord theorem}

\label{sec:tangent_chord_theorem}

A tangent is a line that just touches the circle (or another curve, like a parabola).  By definition it touches the circle at a single point, and is perpendicular to the radius which extends to that point.

$\bullet$  The arc swept out between a tangent and a chord is equal to the arc lying between the point of tangency and the point where the chord meets the circle.

\begin{center} \includegraphics [scale=0.12] {tc_theorem.png} \end{center}

\emph{Proof}.

By Thales' theorem, $\angle B$ is right, so $\angle BAC$ and $\angle BCA$ are complementary.

The tangent $DA \perp AC$, so $\angle BAC$ is complementary to $\angle DAB$.

It follows that $\angle DAB = \angle BCA$.

Therefore, they cut the same arc of the circle.

$\square$

This is Euclid III.32, the tangent-chord theorem.  Acheson calls this the alternate segment theorem.  

Here is Euclid's proof:

In the figure below, let $EF$ be tangent to the circle at $B$ and $AB$ be a diameter of the circle.  $\angle ABF$ is a right angle.

I claim that $\angle DBF = \angle BAD$, and they cut the same arc $DCB$.

Also, $\angle DBE = \angle DCB$, and they cut the same arc $DAB$.

\begin{center} \includegraphics [scale=0.3] {EIII_32_crop.png} \end{center}

\emph{Proof}.

By Thales' theorem $\angle ADB$ is right, so by sum of angles $\angle BAD + \angle ABD$ is right.

Therefore
\[ \angle ABD + \angle BAD = \angle ABD + \angle DBF \]
\[ \angle BAD = \angle DBF \]

For the second part, $\angle DBE + \angle DBF$ equals two right angles.

By the \hyperref[sec:quadrilateral_supplementary]{\textbf{quadrilateral supplementary theorem}} (Euclid III.22)
\[ \angle BAD + \angle BCD = 180 \]

and since $\angle DBF = \angle BAD$ we have
\[ \angle DBE + \angle DBF = \angle DBF + \angle BCD \]
\[ \angle DBE = \angle BCD \]

$\square$

\subsection*{incircle}

\label{sec:incenter}

For any triangle it is possible to draw a circle, called its incircle, which is tangent to all three sides.  When stated in this way, it does not seem obvious how to actually do it.

\begin{center} \includegraphics [scale=0.5] {incircle3.png} \end{center}

\emph{Proof}.

Let $AI$ be the bisector of $\angle A$ and $BI$ be the bisector of $\angle B$, meeting at $I$.

Recall that if we construct the bisector of an angle, say $AI$ bisecting $\angle A$, then every point on the bisector is equidistant from the sides of the original angle.

So one can draw two right triangles with sides $r$ and $AI$, which are congruent by HL, explaining the labels $x$.

The point $I$ is equidistant from side $c$ opposite $\angle C$ and side $b$ opposite $\angle B$.

But the same thing can be done for $BI$ bisecting $\angle B$.

Then draw the circle on center $I$ with radius $r$, and we're done.  This circle is the incircle for $\triangle ABC$.

$r$ is a radius and at the point where it touches the side, perpendicular.  Thus the three sides are each tangent to the incircle.

$\square$

We can obtain a simple result about the area of the triangle.  It is the sum of two copies of each of the three types of right triangle:  $\mathcal{A} = rx + ry + rz$.  But the perimeter of the triangle is twice $x + y + z$, the semi-perimeter $s = x + y + z$ and then $\mathcal{A} = rs$.


\subsection*{Pythagoras incircle proof}

\label{sec:PProof_incircle}

Dunham gives this as a problem.

Let right $\triangle ABC$ have sides opposite $a$, $b$ and $c$, as usual.

Let the parts of the sides be $x$ and $y$.  Note that since this is a right triangle, what would be $z$ becomes $r$, the radius of the incircle.  Also, $c = x + y$ is the hypotenuse. 

\begin{center} \includegraphics [scale=0.35] {pyth25.png} \end{center}

Walk around the triangle.  The perimeter $p$ is
\[ y + x + x + r + r + y \]
\[ p = 2x + 2y + 2r \]

But $x + y = c$ and $p = a + b + c$ so
\[ a + b + c = 2c + 2r \]
\[ 2r = a + b - c \]

Now, the area of the whole $\triangle ABC$ is
\[ K = \frac{1}{2} ab \]

From its three components, the area is also
\[ K = r^2 + rx + ry \]
\[ = r^2 + rc \]
Hence
\[ \frac{1}{2} ab = r^2 + rc \]
\[ 2ab = (2r)^2 + 4rc \]

Substitute for $2r$ and crank through some algebra:
\[ 2ab = (a + b - c )(a + b - c) + 2(a + b - c)(c)  \]
\[ = a^2 + b^2 + c^2 + 2ab - 2ac - 2bc +  2ac + 2bc - 2c^2 \]

Hence
\[ 0 = a^2 + b^2 + c^2 - 2ac - 2bc +  2ac + 2bc - 2c^2 \]
\[ 0 = a^2 + b^2 + c^2 - 2c^2 \]
\[ 0 = a^2 + b^2 - c^2 \]
\[ a^2 + b^2 = c^2 \]

$\square$

Not exactly pretty, but a clever construct, and it works.


\subsection*{Euclid III.12}

\label{sec:Euclid_III_12}

Two circles are tangent to each other at $T$.  

In one circle draw the radius to $T$ and extend it.  This line goes through the center of the second circle.

\begin{center} \includegraphics [scale=0.15] {3pts_tangentc.png} \end{center}

\emph{Proof}.

Draw the tangent to the first circle at $T$.  This line goes through a single point ($T$) on both circles.

The line perpendicular to it (the extension through $T$) is a radius of the second circle.

$\square$

Euclid's proof of III.12 is by contradiction.

\emph{Proof}.

Suppose the straight line $OQ$ does not go through $T$.

By the triangle inequality, $OQ < OT + QT$.
where $OT$ and $QT$ are radii of the two circles.

Then $OQ$ must cut the two circles at different points, say $E$ and $F$.  

The straight line is
\[ OQ = OE + EF + QF \]
\[ OQ > OE + QF \]
where $OE$ and $QF$ are radii of the two circles.

\begin{center} \includegraphics [scale=0.15] {3pts_tangente.png} \end{center}

But $OT + QT = OE + QF$

$OQ$ cannot be both less than the sum of radii and greater than the same sum.  This is a contradiction.

Therefore $OTQ$ are collinear.  $OQ$ goes through $T$.

$\square$

\subsection*{problem}
Draw $AT$ and $BT$.  Prove that $\angle ATB$ is a right angle.

\emph{Solution}.

Draw the tangent line to both circles at $T$, the red line in the figure below.

\begin{center} \includegraphics [scale=0.15] {3pts_tangentd.png} \end{center}

Draw radii of both circles to their respective tangent points at $A$ and $B$.

$DA$ and $DT$ are both tangent to the circle on center $O$, while $DT$ and $DB$ are both tangent to the circle on center $Q$

The tangents are all equal in length, so $\triangle DAT$ and $\triangle DBT$ are both isosceles.

Since the base angles are equal, we have (by the external angle theorem) that
\[ 2 \angle ATD = \angle BDT \]
\[ 2 \angle BTD = \angle ADT \]

But the sum of the two right-hand sides is equal to two right angles.

Therefore, the sum of the two left-hand sides is also equal to two right angles.
Hence
\[  \angle ATD + \angle BTD = 90 = \angle ATB \]

$\square$

Alternatively, since $\triangle DAT$ and $\triangle OAT$ are both isosceles, the sum of the angles making up $\angle OTD$ is equal to the sum of angles making up $\angle OAD$.

But $\angle OAD$ is a right angle, therefore so is $\angle OTD$.

\subsection*{tangents}
As indicated in the previous problem, if two circles just touch one another at a single point, and the tangent is drawn at that point, then it is tangent to \emph{both} circles.  The perpendicular to the tangent at that point is a radius of both circles.  Thus a single line is collinear with both diameters.

\begin{center} \includegraphics [scale=0.25] {circles.png} \end{center}

So then, what about three circles?  Clearly, the third circle can be drawn with the joined diameters of the smaller two as its diameter.  Or it can be drawn somewhat larger (right panel).  

The diameter through $P$ and $Q$ in the second case must be extended to meet the large circle, so apparently, the arrangement in the left panel is the circle with the \emph{smallest} diameter that is also tangent to the two circles on centers $P$ and $Q$.  (See Euclid III.7 for a proof).

\subsection*{The eyeball theorem}

\begin{center} \includegraphics [scale=0.35] {eyeball6.png} \end{center}

\label{sec:eyeball_theorem}

This problem is from Acheson's Geometry book (Fig 131).  We have two circles with tangents drawn from the center of each circle to the other one.

We are to show that the lengths of the chords formed by the tangents as they exit the originating circle are equal.

\emph{Solution}.

We have labeled a number of points.

\[ \triangle OSQ \cong \triangle OTQ \]
by HL.  So $ \angle OQS = \angle OQT$.  As radii, $CQ = DQ$.  

\begin{center} \includegraphics [scale=0.35] {eyeball6.png} \end{center}

Thus $\triangle CXQ \cong \triangle DXQ$ by SAS.  They are two right triangles and have $CX = DX$.  If the bisector of a chord goes through the circle center it is perpendicular.

Let the distance between centers be $d$.  Then we have similar triangles such as
\[ \triangle OSQ \sim \triangle CXQ \]

Let $CX = x$ and the radii be $R$ and $r$.  Similar ratios gives:
\[ \frac{x}{r} = \frac{R}{d} \]

Rearranging $x = Rr/d$.  But this is symmetric in $R$ and $r$ which implies that  $AB = CD$, since we have the same formula for both.

\subsection*{penny-farthing problem problem}

Here is a problem with a fascinating history (see Acheson).  Find an expression for $D$ in terms of $a$ and $b$.

\begin{center} \includegraphics [scale=0.5] {tangent10.png} \end{center}

Its solution is very easy, so I will not give it here but I encourage you to try.  

\subsection*{problem}

Harvard 1899 exam:
\begin{center} \includegraphics [scale=0.5] {Harvard1899_4.png}  \end{center}

\emph{Solution}.

The tangent lines are perpendicular to a radius drawn to the point of tangency.  As a result, two corresponding radii meet at the point of tangency and the two corresponding centers are co-linear with the point of tangency.

\begin{center} \includegraphics [scale=0.4] {3circles_clip.png}  \end{center}

Therefore, we have a triangle with sides as shown. The tangent lines are perpendicular to the sides of the triangle, but will not, in general, pass through a vertex or center of the third circle, for any pair of circles and their tangent line.

But for each vertex of the triangle, the bisector of the angle gives congruent triangles, in which one of the sides is a blue dotted line, forming a right angle with the radius.  The hypotenuse of one such pair is shown in magenta.

The two triangles that share the magenta line as hypotenuse are congruent by hypotenuse-leg in a right triangle (HL), and therefore the two blue dotted lines meeting in the center have equal length.  Now do the same for the circle with radius $r$ and then for the circle with radius $\rho$.

The blue-dotted lines are thus all equal.  So they can be used to draw a circle that just touches the sides of the triangle.  That circle is called the incircle of the triangle.

The point where these tangents meet is the incenter of the triangle.  Since the incenter exists, the three angle bisectors are concurrent, the point where the tangents meet is the same and it also exists.

$\square$

\subsection*{problem}

Here is a problem from Paul Yiu.  We have a semicircle inside a square with one side of the square as the diameter.  The tangent is drawn as shown.

\begin{center} \includegraphics [scale=0.65] {pyth22b.png}  \end{center}
Prove that the triangle is a $3-4-5$ right triangle.

Scale the square so that the side has length $s$.  We will use the property that the two tangents to any circle from an exterior point are equal.  Thus, the distance from the $D$t to the point of tangency at $T$ is $s$.  Let the other short tangents have length $x$.

Then Pythagoras says that:

\[ s^2 + (s-x)^2 = (s+x)^2 \]
\[ s^2 = 4sx \]
\[ s = 4x \]

So $CD = DT = 4x$, and $CE = s - x = 3x$.  This is a right triangle with sides in the ratio 3:4 so it is a 3:4:5 right triangle.

\subsection*{double tangents}

\begin{center} \includegraphics [scale=0.35] {tangent14b.png} \end{center}

This is another problem from Paul Yiu.  Let lines be drawn, each one tangent to both of two circles.  There are four such lines.  The diagram above has simplified labels.  In the next one, the points are labeled for reference.
\begin{center} \includegraphics [scale=0.35] {tangent14c.png} \end{center}

Let $PB$ and $PC$ be tangent to both circles.  Furthermore draw the crossed tangents $KM$ and $LN$.  

The four pairs of short tangents have lengths $p$, $q$, $r$ and $s$.  For a single pair originating from the same point, the two of them are equal.

We want to determine the relationships between the lengths $a,b,c,p,q,r,s$.

The tangents crossing at $X$ are labeled $c$ in the first diagram and $SV$ and $TU$ in the second.  These are equal, because they are composed of two tangents to each circle extended from the intersection in the middle at $X$.  

Before going further, let's look at a simplified diagram.
\begin{center} \includegraphics [scale=0.3] {tangent14d.png} \end{center}

If $PX$ is drawn through the center of the circle at $Q$, the angles at $P$ and $X$ are bisected and then $\triangle PKX \cong \triangle PNX$ by ASA.  It follows that $XK = XN$, and since $XS = XT$, we have $SK = TN$.  Similarly, $KA = ND$.

So, looking again at the problem
\begin{center} \includegraphics [scale=0.4] {tangent14b.png} \end{center}
we see that $p = s$ and $q = r$.  The next question is the relationship between $p$ and $q$.  We can write other equalities based on tangents from different points:

\[ LA = p + a = q + c = LT \]
\[ KV = p + c = a + q = KB \]
\begin{center} \includegraphics [scale=0.35] {tangent14c.png} \end{center}

Adding and cancelling $a + c$, we have that $p = q$.
\begin{center} \includegraphics [scale=0.35] {tangent14b.png} \end{center}

So all four of the short segments are also equal:  $p = q = r = s$.  And substituting above, we find that $a = c$.

Finally, we derive the position of $P$, the origin point of the tangents that do not cross.  We also derive the position of $X$, the intersection of the tangents that do cross,

Since both lie on the line between centers, we worry only about the horizontal dimension.  Let $PO = y$.  Let the radius of the circle on center $O$ be $R$ and that of the other circle on $Q$ be $\rho$.

Let the distance between the two circles be $d$.  Then we have two similar right triangles, with
\[ \frac{y}{R} = \frac{y+d}{\rho} \]
\[ \frac{\rho}{R} = \frac{y+d}{y} = 1 + \frac{d}{y} \]
which is easily solved for $y$.

\[ y = \frac{d}{\rho/R - 1} \]

To find $OX = x$, we have a closely related equation 
\[ \frac{x}{R} = \frac{d-x}{\rho} \]
which is worked out by similar algebra.

\[ y = \frac{d}{\rho/R + 1} \]

\end{document}