\documentclass[11pt, oneside]{article} 
\usepackage{geometry}
\geometry{letterpaper} 
\usepackage{graphicx}
	
\usepackage{amssymb}
\usepackage{amsmath}
\usepackage{parskip}
\usepackage{color}
\usepackage{hyperref}

\graphicspath{{/Users/telliott/Github-Math/figures/}}
% \begin{center} \includegraphics [scale=0.4] {gauss3.png} \end{center}

\title{Cyclic Quadrilateral}
\date{}

\begin{document}
\maketitle
\Large

%[my-super-duper-separator]

\label{sec:quadrilateral_supplementary}

There is a wonderfully simple theorem about quadrilaterals (Euclid III.22).  A cyclic quadrilateral is a four-sided polygon whose four vertices all lie on one circle.

\begin{center} \includegraphics [scale=0.12] {EIII_22.png} \end{center}

$\bullet$ \ For \emph{any} cyclic quadrilateral, the opposing angles are supplementary (they sum to two right angles).

\emph{Proof}.

Together, opposing angles in a cyclic quadrilateral exactly correspond to the whole arc of the circle.

Since the central angle for that arc is four right angles, the sum of opposing inscribed angles is just one-half that or two right angles.

$\square$

Euclid's proof uses sum of angles:

\begin{center} \includegraphics [scale=0.12] {EIII_22.png} \end{center}

\emph{Proof}.

$\angle ADC = \angle ADB + \angle BDC$, subtended by arcs $AB$ and $BC$.

As angles on equal arcs (III.21) the latter two angles are equal to $\angle ACB$ and $\angle BAC$.

But by I.32, the same two angles plus the total angle at vertex $B$ are equal to two right angles.

$\square$

\subsection*{converse of cyclic quadrilateral theorem}

This one uses Euclid III.22 rather than III.21.  

\label{sec:inscribed_angles_converse2}

Let $\triangle ABC$ lie on a circle.  

Let point $D$ be such that $\angle ADC$ is supplementary to $\angle ABC$.

Then $D$ lies on the same circle.
\begin{center} \includegraphics [scale=0.22] {cyclic_quad_converse_b} \end{center}

\emph{Proof}.

Aiming for a contradiction, suppose $D$ does not lie on the circle.

Let $D$ be external and $D'$ lie on the point where $BD$ cuts the circle (right panel).  

By the forward theorem, $\angle AD'C$ is supplementary to $\angle ABC$ and so equal to $\angle ADC$.

But by Euclid I.21,  $\angle ADC < \angle AD'C$.  

This is a contradiction.  Therefore $D$ is not external.  

A similar argument will show that $D$ is not internal.  

Therefore $D$ lies on the circle and $D$ and $D'$ are the same point.

$\square$

\subsection*{bisected angles and SSA}

Take an arbitrary triangle and draw its circumcircle.  Bisect one angle say $\angle A$, and find where the bisector cuts the circle at $D$.

\begin{center} \includegraphics [scale=0.3] {cq_SSA_crop} \end{center}

We notice that since the arcs are equal, $CD = DB$, and with a shared side, we have SSA for the two triangles:  $\triangle ADC$ and $\triangle ABD$.

These two triangles are obviously \emph{not} congruent.

Yet they are related by SSA and if not congruent, we expect that the angles $\angle ACD$ and $\angle ABD$ are supplementary (see the chapter on congruence tests).

But of course they are, since they are opposing vertices in a cyclic quadrilateral.

We also have that $\triangle BCD$ is isosceles.

If the arc for any of the angles is bisected then the line drawn to that point from the vertex is the angle bisector.

The next problem is related to the previous discussion.

\begin{center} \includegraphics [scale=0.75] {cq_problem.png} \end{center}

Given that $ABDC$ is a cyclic quadrilateral.  Let $BD$ bisect $\angle CBE$.  Prove that $DC = DA$.

\emph{Proof}.

Let $\beta$ be one-half the bisected external angle so $\beta = \angle DBE = \angle DBC$.

$\angle DAC$ is on arc $CD$ so $\angle DAC = \beta$ by inscribed angles on equal arcs.

$\angle DCA$ is supplementary to $\angle ABD$ by the fundamental theorem of the cyclic quadrilateral.

And then $\angle ABD$ is supplementary to $\beta$ on the straight line $ABE$.

So $\angle DCA = \beta = \angle DAC$.

It follows that $\triangle DCA$ is isosceles, with $DC = DA$.

$\square$


\subsection*{Brahmagupta's theorem}

I have redrawn the figure from wikipedia.

\url{https://en.wikipedia.org/wiki/Brahmagupta_theorem}

\begin{center} \includegraphics [scale=0.75] {bg6.png} \end{center}

$ABCD$ is a cyclic quadrilateral with diagonals that cross at right angles.

Given $NM \perp AB$.  Show that $CK$ is bisected, with $CN = NK$.

If the conclusion is true, then since $\triangle CDK$ is right, we should have that the median $ND$ is equal to the two halves of $CK$.  This suggests we try to prove that the two smaller triangles are isosceles.

\emph{Proof}.

The perpendicular $DM$ in right triangle $BDA$ forms two smaller similar right triangles:  $\triangle BDA \sim \triangle DMA \sim \triangle MBD$.

In $\triangle DKN$, $\angle DKN = \angle ABC$ by inscribed angles on equal arcs.  

But $\angle ABC = \angle ADM$ by the similar triangles just given, and then $\angle ADM = \angle NDK$ by vertical angles.

Thus $\triangle NDK$ is isosceles, with $NK = ND$.

For the same reason, $\triangle CND$ is also isosceles.  So then

\[ CN = ND = NK \]

$\square$

\subsection*{Van Schooten's theorem}

\begin{center} \includegraphics [scale=0.45] {Van_Schooten.png} \end{center}

This is given as a problem by Surowski (1.3.6).

Given an equilateral $\triangle ABC$ draw its circumcircle.

Draw an arbitrary line segment from vertex $A$ through side $BC$ to meet the circle at $M$.  Prove that $AM = BM + MC$.

\emph{Proof}.

Draw lines from $M$ to each vertex of $\triangle ABC$ and extend $MB$ to give $MD = AM$.  Join $AD$.

By inscribed angles both angles at $M$ are equal to two-thirds of a right angle.

Since $MD = AM$, $\triangle AMD$ is isosceles with the vertex equal to two-thirds of a right angle.

Thus $\triangle AMD$ is equilateral.   This accounts for all of the red dots.  ($\angle A$ is not marked, to keep the diagram simpler).

Subtract the central $\angle MAB$ from two equal angles to yield $\angle CAM = \angle BAD$.

\begin{center} \includegraphics [scale=0.45] {Van_Schooten.png} \end{center}

$\triangle CAM \sim \triangle BAD$.

But $AB = AC$ and also $AM = AD$, so $\triangle BAD \cong \triangle CAM$ by ASA or SAS.

It follows that $BD = MC$.

Thus $BD + MB = MC + MB$ and the former is equal to MD.

$\square$

This theorem is easy to prove using Ptolemy's theorem.  Ptolemy says that, in a cyclic quadrilateral, multiply the lengths of opposing sides and add the two products, and that is equal to the product of the diagonals.

Here, let $AB = AC = BC = 1$.  Then Ptolemy says:
\[ MC \cdot 1 + MB \cdot 1 = AM \cdot 1 \]
\[ MC + MB = AM \]

However, we don't have that quite yet.  So we did it as suggested.

\subsection*{we have questions}

I saw this question on the internet:  ``is a parallelogram a cyclic quadrilateral?''

In a parallelogram, opposing angles are equal, while in a cyclic quadrilateral, opposing angles are supplementary.

The only supplementary, equal angles are two right angles.  So a rectangle is the only parallelogram that is a cyclic quadrilateral.



\end{document}