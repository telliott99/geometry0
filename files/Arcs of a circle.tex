\documentclass[11pt, oneside]{article} 
\usepackage{geometry}
\geometry{letterpaper} 
\usepackage{graphicx}
	
\usepackage{amssymb}
\usepackage{amsmath}
\usepackage{parskip}
\usepackage{color}
\usepackage{hyperref}

\graphicspath{{/Users/telliott/Dropbox/Github-Math/figures/}}
% \begin{center} \includegraphics [scale=0.4] {gauss3.png} \end{center}

\title{Arcs of a circle}
\date{}

\begin{document}
\maketitle
\Large

%[my-super-duper-separator]

It is natural to think about angles formed from a vertex lying on the periphery of a circle and ask about their relation to the arcs they cut off.

The \hyperref[sec:inscribed_angle_theorem]{\textbf{inscribed angle theorem}} says that a vertex placed at any point on the periphery of a circle, forming an angle that corresponds to the same arc as a central angle, is equal to \emph{one-half} the central angle.

Since the measure of the central angle is equal to the measure of the arc, by definition, we have that twice the peripheral angle is equal to the arc that subtends it.

A corollary is that whenever two peripheral angles correspond to the same arc, they are equal (\hyperref[sec:angles_on_same_arc]{\textbf{equal angles $\iff$ equal arcs}}).

In the figure below, the two angles marked $\alpha$ are equal, because they correspond to the same arc of the same circle.
\begin{center} \includegraphics [scale=0.35] {arcs_2.png} \end{center}

Similarly, the two angles marked $\beta$ are equal, for the same reason.  Then by sum of angles (or vertical angles), the two angles marked $\gamma$ are also equal.  We have two similar triangles.

If two chords of the circle cross, the product of the components is a constant.

\emph{Proof}.

If we label the sides of the triangles
\begin{center} \includegraphics [scale=0.35] {arcs_3.png} \end{center}
$u$ and $t$ are the sides opposite $\angle \alpha$ so we have
\[ \frac{u}{t} = \frac{s}{v} \]
rearranging
\[ u \cdot v = s \cdot t \]

$\square$

This is the  \hyperref[sec:chord_segments]{\textbf{crossed chord theorem}} (product of lengths).

\subsection*{Arcs of intersecting chords}
\begin{center} \includegraphics [scale=0.3] {arcs_4.png} \end{center}

Given two crossed chords, $\theta$ is the average of the opposing arcs $a$ and $b$.  

\emph{Proof}.

Take the same two similar triangles and consider the angle $\theta$ as shown in the figure above.  By the corollary of the inscribed angle theorem, we have that
\[ 2 \alpha = a \ \ \ \ \ \ \ \ 2 \beta = b \]

But $\theta$ is the external angle to both triangles, so $\theta = \alpha + \beta$ and then
\[ 2 \theta = 2 \alpha + 2 \beta = a + b \]
\[ \theta = \frac{a + b}{2} \]

$\square$

\subsection*{external vertex}

Rather than having the vertex on the circle, suppose it lies outside.

\begin{center} \includegraphics [scale=0.3] {arcs_5.png} \end{center}

If the angle lies outside the circle, then twice its measure is the difference between the two arcs, the one farther away minus the closer one.

\emph{Proof}.

Again, $\alpha$ corresponds to arc $a$ and $\beta$ to arc $b$.  
\[ 2 \alpha = a \ \ \ \ \ \ \ \ 2 \beta = b \]

But $\beta$ is the external angle to the small triangle with $\theta$, such that
\[ \beta = \theta + \alpha \]
\[ 2 \theta = 2 \beta - 2 \alpha \]
\[ 2 \theta = b - a \]
\[ \theta = \frac{b-a}{2} \]

$\square$

\subsection*{tangent and secant}

\begin{center} \includegraphics [scale=0.3] {arcs_6.png} \end{center}

Rather than two secants, we now have a secant and a tangent.  The result is the same as previously.

\[ \theta = \frac{b - a}{2} \]

\emph{Proof}.

We rely on the tangent-chord theorem, which says that if $\alpha$ is the angle between a chord and a tangent, then the corresponding arc has the same relationship as for two chords emanating from the vertex, namely:
\[ 2 \alpha = a \]
\[ 2 \beta = b \]

But
\[ \beta = \alpha + \theta \]
\[ 2 \theta = 2 \beta - 2 \alpha \]
\[ 2 \theta = b - a \]
\[ \theta = \frac{b-a}{2} \]

$\square$

\subsection*{two tangents}

We showed previously that when two tangents are drawn from an exterior point, one can draw two right triangles that share the hypotenuse and have another side equal to the radius, so they are congruent by hypotenuse-leg in a right triangle (HL).

\begin{center} \includegraphics [scale=0.35] {tangent_arcs.png} \end{center}

Let the whole arc between the two right angles be $s$ the short way and $t$ the long way around the circle, and let $\phi$ be the external angle.  By analogy with the results above, we expect that 
\[ \phi = \frac{t - s}{2} \]

\emph{Proof}.

We could use congruent triangles, but instead just note that the sum of angles in any quadrilateral is equal to four right angles.  Hence
\[ \phi + \theta = \ \text{two right angles}  \]

In terms of arc $\theta = s$ and $s + t =$ two right angles. Substituting into the last equation
\[ \phi + s = \frac{s + t}{2} \]
\[ \phi = \frac{t - s}{2} \]

$\square$

\subsection*{problem}

Relate the angle at $P$ to the one at $X$.

\begin{center} \includegraphics [scale=0.35] {tangent_arcs2.png} \end{center}

By the previous example, $\theta + \phi = 180$.  But $\angle X = \theta/2$.  Hence
\[ P = 180  - \theta = 180 - 2 \angle X \]

\subsection*{problem}

\begin{center} \includegraphics [scale=0.2] {broken_chord17b.png} \end{center}

Let $AB$ be a diameter of the circle on center $O$.  Let $M$ be found such that $AM = BM$, so $\angle MAB = \angle MBA$ and both are one-half of a right angle.

Draw an arbitrary chord from $A$ such as $AC$.  Draw $MP \perp AC$.

$\triangle MPC$ is isosceles.

\emph{Proof}.

Draw $MC$.

$\angle ACM = \angle PCM$ is subtended by chord $AM$, which is one-quarter of the circle.

Since $\angle MPC$ is right, $\angle PMC$ is also one-half of a right angle, by sum of angles.

It follows that $\triangle MPC$ is isosceles and $MP = CP$.

$\square$

\end{document}